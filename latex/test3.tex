%% Generated by Sphinx.
\def\sphinxdocclass{report}
\documentclass[letterpaper,10pt,english]{sphinxmanual}
\ifdefined\pdfpxdimen
   \let\sphinxpxdimen\pdfpxdimen\else\newdimen\sphinxpxdimen
\fi \sphinxpxdimen=.75bp\relax
\ifdefined\pdfimageresolution
    \pdfimageresolution= \numexpr \dimexpr1in\relax/\sphinxpxdimen\relax
\fi
%% let collapsable pdf bookmarks panel have high depth per default
\PassOptionsToPackage{bookmarksdepth=5}{hyperref}

\PassOptionsToPackage{warn}{textcomp}
\usepackage[utf8]{inputenc}
\ifdefined\DeclareUnicodeCharacter
% support both utf8 and utf8x syntaxes
  \ifdefined\DeclareUnicodeCharacterAsOptional
    \def\sphinxDUC#1{\DeclareUnicodeCharacter{"#1}}
  \else
    \let\sphinxDUC\DeclareUnicodeCharacter
  \fi
  \sphinxDUC{00A0}{\nobreakspace}
  \sphinxDUC{2500}{\sphinxunichar{2500}}
  \sphinxDUC{2502}{\sphinxunichar{2502}}
  \sphinxDUC{2514}{\sphinxunichar{2514}}
  \sphinxDUC{251C}{\sphinxunichar{251C}}
  \sphinxDUC{2572}{\textbackslash}
\fi
\usepackage{cmap}
\usepackage[T1]{fontenc}
\usepackage{amsmath,amssymb,amstext}
\usepackage{babel}



\usepackage{tgtermes}
\usepackage{tgheros}
\renewcommand{\ttdefault}{txtt}



\usepackage[Bjarne]{fncychap}
\usepackage{sphinx}

\fvset{fontsize=auto}
\usepackage{geometry}


% Include hyperref last.
\usepackage{hyperref}
% Fix anchor placement for figures with captions.
\usepackage{hypcap}% it must be loaded after hyperref.
% Set up styles of URL: it should be placed after hyperref.
\urlstyle{same}

\addto\captionsenglish{\renewcommand{\contentsname}{Installtion Guide}}

\usepackage{sphinxmessages}
\setcounter{tocdepth}{1}



\title{test3 Documentation}
\date{Jun 04, 2021}
\release{1}
\author{test3}
\newcommand{\sphinxlogo}{\vbox{}}
\renewcommand{\releasename}{Release}
\makeindex
\begin{document}

\pagestyle{empty}
\sphinxmaketitle
\pagestyle{plain}
\sphinxtableofcontents
\pagestyle{normal}
\phantomsection\label{\detokenize{index::doc}}



\chapter{Linux Information}
\label{\detokenize{index:linux-information}}
\sphinxAtStartPar
Once upon a time, before Linux there were some Operating Systems. This was known as a dark age of server operating systems.
Now there is Linux.

\sphinxAtStartPar
This documentation demo doesn’t do or achieve much, except for illustrating how documentation could look.

\sphinxAtStartPar
Please fasten seatbeats, and enjoy the ride.


\section{Guide}
\label{\detokenize{index:guide}}
\sphinxAtStartPar
Contents:


\subsection{Installation Documentation}
\label{\detokenize{install:installation-documentation}}\label{\detokenize{install::doc}}

\subsubsection{Introduction}
\label{\detokenize{install:introduction}}
\sphinxAtStartPar
This section describes the Linux Installation


\subsubsection{Check Ansible Server}
\label{\detokenize{install:check-ansible-server}}

\paragraph{Log onto Ansible}
\label{\detokenize{install:log-onto-ansible}}
\sphinxAtStartPar
Log onto the Ansible Server as an Admin (user with sudo yum) user and issue command

\begin{sphinxVerbatim}[commandchars=\\\{\}]
uname \PYGZhy{}a
cat /etc/redhat\PYGZhy{}release
\end{sphinxVerbatim}

\sphinxAtStartPar
The above will show the current linux version


\paragraph{Check yum repos}
\label{\detokenize{install:check-yum-repos}}
\begin{sphinxVerbatim}[commandchars=\\\{\}]
yum repolist
\end{sphinxVerbatim}

\sphinxAtStartPar
Ensure that there are no repos with 0 items


\paragraph{Configure Variables}
\label{\detokenize{install:configure-variables}}
\sphinxAtStartPar
Define additional variables

\sphinxAtStartPar
There should be an additional variables that need to be defined

\begin{sphinxVerbatim}[commandchars=\\\{\}]
\PYG{n+nb}{export} \PYG{n+nv}{DUMMYVAR}\PYG{o}{=}foo
\end{sphinxVerbatim}


\subsubsection{Installation via Command Line}
\label{\detokenize{install:installation-via-command-line}}

\paragraph{Run Job}
\label{\detokenize{install:run-job}}
\sphinxAtStartPar
Issue the command, press \sphinxcode{\sphinxupquote{y}} and then \sphinxcode{\sphinxupquote{{[}ENTER{]}}} when prompted

\begin{sphinxVerbatim}[commandchars=\\\{\}]
sudo yum update
\end{sphinxVerbatim}

\begin{sphinxadmonition}{warning}{Warning:}
\sphinxAtStartPar
This command may take some time
\end{sphinxadmonition}


\paragraph{Reboot}
\label{\detokenize{install:reboot}}
\sphinxAtStartPar
If kernel packages were updated, issue the following command:

\begin{sphinxVerbatim}[commandchars=\\\{\}]
sudo reboot
\end{sphinxVerbatim}

\sphinxAtStartPar
\sphinxcode{\sphinxupquote{Note:}} Confirm server reboots


\subsection{Legacy Install Document}
\label{\detokenize{legacy/legacy-install:legacy-install-document}}\label{\detokenize{legacy/legacy-install::doc}}

\subsubsection{Introduction}
\label{\detokenize{legacy/legacy-install:introduction}}
\sphinxAtStartPar
This section describes the Linux Installation


\subsubsection{Check Compile Server}
\label{\detokenize{legacy/legacy-install:check-compile-server}}

\paragraph{Log onto Compile Server}
\label{\detokenize{legacy/legacy-install:log-onto-compile-server}}
\sphinxAtStartPar
Log onto the Compile Server as an Admin (user with sudo yum) user and issue command

\begin{sphinxVerbatim}[commandchars=\\\{\}]
uname \PYGZhy{}a
cat /etc/redhat\PYGZhy{}release
\end{sphinxVerbatim}

\sphinxAtStartPar
The above will show the current linux version


\paragraph{Check yum repos}
\label{\detokenize{legacy/legacy-install:check-yum-repos}}
\begin{sphinxVerbatim}[commandchars=\\\{\}]
yum repolist
\end{sphinxVerbatim}

\sphinxAtStartPar
Ensure that there are no repos with 0 items


\paragraph{Configure Variables}
\label{\detokenize{legacy/legacy-install:configure-variables}}
\sphinxAtStartPar
Define additional variables

\sphinxAtStartPar
There should be an additional variables that need to be defined

\begin{sphinxVerbatim}[commandchars=\\\{\}]
\PYG{n+nb}{export} \PYG{n+nv}{DUMMYVAR}\PYG{o}{=}foo
\end{sphinxVerbatim}


\subsubsection{Installation via Command Line}
\label{\detokenize{legacy/legacy-install:installation-via-command-line}}

\paragraph{Run Job}
\label{\detokenize{legacy/legacy-install:run-job}}
\sphinxAtStartPar
Issue the command, press \sphinxcode{\sphinxupquote{y}} and then \sphinxcode{\sphinxupquote{{[}ENTER{]}}} when prompted

\begin{sphinxVerbatim}[commandchars=\\\{\}]
sudo yum update
\end{sphinxVerbatim}

\begin{sphinxadmonition}{warning}{Warning:}
\sphinxAtStartPar
This command may take some time
\end{sphinxadmonition}


\paragraph{Reboot}
\label{\detokenize{legacy/legacy-install:reboot}}
\sphinxAtStartPar
If kernel packages were updated, issue the following command:

\begin{sphinxVerbatim}[commandchars=\\\{\}]
sudo reboot
\end{sphinxVerbatim}

\sphinxAtStartPar
\sphinxcode{\sphinxupquote{Note:}} Confirm server reboots


\subsection{Procedures Documentation}
\label{\detokenize{procedures:procedures-documentation}}\label{\detokenize{procedures::doc}}

\subsubsection{Introduction}
\label{\detokenize{procedures:introduction}}
\sphinxAtStartPar
This section describes the a demo Procedure


\subsubsection{Check Ansible Version}
\label{\detokenize{procedures:check-ansible-version}}

\paragraph{Log onto Ansible}
\label{\detokenize{procedures:log-onto-ansible}}
\sphinxAtStartPar
Log onto the Ansible Server as an Admin user and issue command

\begin{sphinxVerbatim}[commandchars=\\\{\}]
ansible \PYGZhy{}\PYGZhy{}version
\end{sphinxVerbatim}

\sphinxAtStartPar
The above will show the current Ansible version


\subsubsection{Update Ansible Inventory}
\label{\detokenize{procedures:update-ansible-inventory}}
\sphinxAtStartPar
The Ansible inventory uses a YAML formatted file, therefore spacing is very important.


\paragraph{Update Inventory}
\label{\detokenize{procedures:update-inventory}}
\sphinxAtStartPar
Add the following block to the \sphinxcode{\sphinxupquote{/etc/ansible/hosts}} files

\begin{sphinxVerbatim}[commandchars=\\\{\}]
\PYG{o}{[}webserver\PYG{o}{]}
servera.demo.net
\end{sphinxVerbatim}


\chapter{Indices and tables}
\label{\detokenize{index:indices-and-tables}}\begin{itemize}
\item {} 
\sphinxAtStartPar
\DUrole{xref,std,std-ref}{genindex}

\item {} 
\sphinxAtStartPar
\DUrole{xref,std,std-ref}{modindex}

\item {} 
\sphinxAtStartPar
\DUrole{xref,std,std-ref}{search}

\end{itemize}

\begin{sphinxadmonition}{note}{Note:}
\sphinxAtStartPar
You can get a PDF copy of this document here: \sphinxurl{https://linuxbuff.github.io/test3/latex/test3.pdf}
\end{sphinxadmonition}



\renewcommand{\indexname}{Index}
\printindex
\end{document}